\documentclass[dvipsnames]{beamer}

%\usepackage{fontspec}                           	% Typo
\usepackage[french,english]{babel}
\usepackage{empheq}                             	% Encadrer les équations
\usepackage{amsthm}                             	% Démonstration de maths
\usepackage{tabularx}                           	% Tableaux avancés
\usepackage{fancyvrb}                           	% Blocs mono avancés
\usepackage{caption}                            	% Légendes avancées
\usepackage{tikz}                               	% Outil de dessin
\usepackage{listings}                           	% Coloration syntaxique
\usepackage{xcolor}                             	% Color support
\usepackage{enumerate}
\usetikzlibrary{calc,arrows,shapes.misc}
\usepackage{hyperref}

\tikzset{cross/.style={cross out, draw=black, minimum size=8*(#1-\pgflinewidth), inner sep=0pt, outer sep=0pt},cross/.default={1pt}}
\definecolor{salmon}{HTML}{FA8072}
\usefonttheme[onlymath]{serif} % Force les maths en serif

%%%%%%%%%%%%%%%%%%%%%
\usetheme{Frankfurt}	
\usecolortheme[rgb={0,0.6,0}]{structure}
\captionsetup{font=scriptsize,labelfont=scriptsize}
\setbeamertemplate{caption}{\raggedright\insertcaption\par}
%%%%%%%%%%%%%%%%%%%%%

\begin{document}


% Agmente les marges entre paragraphes % \setlength{\parskip}{1em}
\title{Introduction aux Bases de Données}

% Numerotations
\addtobeamertemplate{navigation symbols}{} { \usebeamerfont{footline} \usebeamercolor[fg]{footline} \hspace{1em} \insertframenumber/\inserttotalframenumber }

\author{Ulysse COUTAUD}

\date{}
	
\maketitle


\section{Editer le schéma}


\begin{frame}{Créer une table}
	\begin{alertblock}{}
		
	\end{alertblock}
\end{frame}

\begin{frame}{Ajouter une colonne}
	\begin{alertblock}{}
		
	\end{alertblock}
\end{frame}

\begin{frame}{Supprimer une colonne}
	\begin{alertblock}{}
		
	\end{alertblock}
\end{frame}

\begin{frame}{Modifier une colonne}
	\begin{alertblock}{}
		
	\end{alertblock}
\end{frame}




\end{document}
