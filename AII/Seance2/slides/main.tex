\documentclass[xetex,dvipsnames]{beamer}

%\usepackage{fontspec}                           	% Typo
\usepackage[french,english]{babel}
\usepackage{empheq}                             	% Encadrer les équations
\usepackage{amsthm}                             	% Démonstration de maths
\usepackage{tabularx}                           	% Tableaux avancés
\usepackage{fancyvrb}                           	% Blocs mono avancés
\usepackage{caption}                            	% Légendes avancées
\usepackage{tikz}                               	% Outil de dessin
\usepackage{listings}                           	% Coloration syntaxique
\usepackage{xcolor}                             	% Color support
\usepackage{enumerate}
\usetikzlibrary{calc,arrows,shapes.misc}
\usepackage{hyperref}


\tikzset{cross/.style={cross out, draw=black, minimum size=8*(#1-\pgflinewidth), inner sep=0pt, outer sep=0pt},cross/.default={1pt}}
\definecolor{salmon}{HTML}{FA8072}
\usefonttheme[onlymath]{serif} % Force les maths en serif


%%%%%%%%%%%%%%%%%%%%%
\usetheme{Frankfurt}	
\usecolortheme[rgb={0,0.6,0}]{structure}
\captionsetup{font=scriptsize,labelfont=scriptsize}
\setbeamertemplate{caption}{\raggedright\insertcaption\par}
%%%%%%%%%%%%%%%%%%%%%

\begin{document}


% Agmente les marges entre paragraphes % \setlength{\parskip}{1em}
\title{Introduction aux Bases de Données}

% Numerotations
\addtobeamertemplate{navigation symbols}{} { \usebeamerfont{footline} \usebeamercolor[fg]{footline} \hspace{1em} \insertframenumber/\inserttotalframenumber }

\author{Ulysse COUTAUD\\\href{mailto:ulysse.coutaud@gmail.com}{\small ulysse.coutaud@gmail.com}}
\date{}
	
\maketitle

\section{Modifier des données}
\begin{frame}
\begin{center}
{\LARGE \textbf{Modifier des données}}
\end{center}
\end{frame}

\begin{frame}[t]{Insérer un enregistrement}
\begin{small}
		BDD "atelier": \\employes(\underline{prenom},salaire,adresse,service) horaires(\underline{service},debut,fin)
\end{small}	
	\begin{alertblock}{Clause INSERT INTO}
	INSERT INTO table VALUES ('valeur 1', 'valeur 2', ...)\\
	INSERT INTO table (nom\_colonne\_1, nom\_colonne\_2, ...) VALUES ('valeur 1', 'valeur 2', ...)
	\end{alertblock}
		
	\textit{Ajouter un nouvel employé dans le service "Production"}\\
	\textit{Ajouter un service "Production Après-Midi" dont les horaires seront 13h-20h.}\\
	\textit{Ajouter les horaires 16h-5h pour le service "Astreinte"}\\
	\textit{Ajouter un nouvel employé "Abdoul" au service "Production Après-Midi".}\\
\end{frame}

\begin{frame}{Modifier un enregistrement}
	\begin{small}
		BDD "atelier": \\employes(\underline{prenom},salaire,adresse,service) horaires(\underline{service},debut,fin)
	\end{small}	
	\begin{alertblock}{Clause UPDATE}
		UPDATE table SET colonne\_1 = 'valeur 1', colonne\_2 = 'valeur 2', ... WHERE conditions;
	\end{alertblock}
	\textit{Augmenter le salaire des employes de "Astreinte" de 30\%}\\
	\textit{L'employé Tom change de service et passe à l'après midi }\\
\end{frame}

\begin{frame}{Supprimer un enregistrement}
	\begin{small}
		BDD "atelier": \\employes(\underline{prenom},salaire,adresse,service) horaires(\underline{service},debut,fin)
	\end{small}	
	\begin{alertblock}{Clause DELETE FROM}
		DELETE FROM table WHERE conditions;
	\end{alertblock}
	\textit{Supprimer le service de production après-midi.}\\
\end{frame}

\section{Modifier le schéma}
\begin{frame}
\begin{center}
{\LARGE \textbf{Modifier le schéma}}
\end{center}
\end{frame}

\begin{frame}{Créer une table}
	\begin{alertblock}{CREATE TABLE}
	CREATE TABLE \textit{ma\_table} (\\
    \textit{colonne1}		TEXT,\\
	\textit{colonne2}		INTEGER\\
);
	\end{alertblock}
	\textit{Ajouter une table "ordre\_fabrication" avec 4 colonnes: id (le numéro d'OF), nom (l'employé qui va produire l'OF), reference\_piece (la référence de pièce à produire)}
\end{frame}

\begin{frame}{Supprimer une table}
	\begin{alertblock}{DROP TABLE}
	DROP TABLE \textit{ma\_table};
	\end{alertblock}
	\textit{Supprimer la table horaire.}
\end{frame}


\begin{frame}{Modifier une table}
	\begin{alertblock}{ALTER TABLE}
	ALTER TABLE ma\_table mon\_action ;\\
	Avec mon\_action:
	\begin{itemize}
		\item ADD ma\_colonne INTEGER/TEXT/...
		\item DROP ma\_colonne
	\end{itemize}
	\end{alertblock}
	\textit{Ajouter la colonne quantite (le nombre de pièces à produire) à la table ordre\_fabrication. }
\end{frame}




 
\section{Les contraintes}
\begin{frame}
\begin{center}
{\LARGE \textbf{Les contraintes}}
\end{center}
\pause
\begin{itemize}
	\item Règles qui restreignent les données dans la BDD.\\
	\item Chaque contrainte est associée à une table.\\
	\item Bloque tout ajout/modification qui viole les contraintes.
	\item Garantis \textbf{certains} aspects de la cohérence de la BDD.
\end{itemize}
\end{frame}

\begin{frame}{Clé Primaire}
	\begin{alertblock}{PRIMARY KEY}
		CONSTRAINT \textit{ma\_clé\_primaire} PRIMARY KEY (\textit{colonne1})\\
		Unique et non nulle.\\
		Identifie l'enregistrement/ligne.\\
	\end{alertblock}
	
	\textit{Modifier le nom de Peter en John. Que ce passe il ? }\\
\vspace*{1em}
	Remarque: Si la clé primaire n'est pas définie à la création de la table \pause c'est l'ensemble des colonnes, cad la ligne entière.
	
	
\end{frame}

\begin{frame}{Clé Secondaire}
	\begin{alertblock}{UNIQUE}
		CONSTRAINT \textit{ma\_clé\_secondaire} UNIQUE (\textit{colonne1})\\
		Unique mais peut être nulle.
	\end{alertblock}
	\textit{Avec la commande psql "\\d+ employes", trouver une clé secondaire de la table employes et expliquer sont sens.}\\	
\end{frame}

\begin{frame}{Clé étrangère}
	\begin{alertblock}{FOREIGN KEY}
		CONSTRAINT \textit{ma\_clé\_étrangère} FOREIGN KEY (colonne1) REFERENCES table\_etrangère(colonneX))\\
		Assure l'intégrité référentielle.\\
		Existe nécessairement dans la table de référence.
	\end{alertblock}
	\begin{block}{Et si la clé de référence disparait ?}
		Trois options:
		\begin{itemize}
			\item ON DELETE RESTRICT
			\item ON DELETE CASCADE
			\item ON DELETE SET NULL
		\end{itemize}				
	\end{block}
\end{frame}


\begin{frame}{Check}
	\begin{alertblock}{CHECK}
		CONSTRAINT \textit{mon\_check} CHECK (condition)\\
	\end{alertblock}
	Remarque: Même genre de conditions que dans un WHERE.
	\vspace*{1em}
	\textit{En lisant le fichier fabriqueBDD.sql: quelle contrainte CHECK existe sur la table employes ?}\\
	\textit{Quelle contrainte devrait on ajouter sur la colonne salaire ?}\\
\end{frame}




\section{Les transactions}
\begin{frame}
\begin{center}
{\LARGE \textbf{Les transactions}}
\end{center}
\end{frame}

\begin{frame}[fragile]{Le problème}
	\begin{verbatim}
UPDATE accounts SET balance = balance - 100.00
    WHERE name = 'Alice';
    
UPDATE branches SET balance = balance - 100.00
    WHERE name = (SELECT branch_name 
                  FROM accounts 
                  WHERE name = 'Alice');
                 
UPDATE accounts SET balance = balance + 100.00
    WHERE name = 'Bob';
    
UPDATE branches SET balance = balance + 100.00
    WHERE name = (SELECT branch_name 
                  FROM accounts 
                  WHERE name = 'Bob');
    \end{verbatim}
\end{frame}

\begin{frame}[fragile]{La solution}
	\begin{verbatim}
BEGIN;
UPDATE accounts SET balance = balance - 100.00
    WHERE name = 'Alice';
-- etc etc
COMMIT;
    \end{verbatim}
    Le SGBDR garantis des transactions ACID:
	\begin{itemize}
		\item \textbf{A}tomicité
		\item \textbf{C}ohérence
		\item \textbf{I}solation
		\item \textbf{D}urabilité
	\end{itemize}
    
\end{frame}


\end{document}
